\subsubsection{\textsc{Teil I -- Hoffnungen}}
\emph{Szene möglichst Klischeehaft, Spannung aufbauend, wie in einem\note{Prolog}
Jugendkrimi oder sowas.}
Quinn, eine \qq{ganz normale} junge Erwachsene in der
spät-ka\-pi\-ta\-lis\-tischen Welt, geschieht eines Morgens etwas Seltsames.
Am U-Bahnhof \emph{Kirchplatz} in Frankfurt am Main wird ihr plötzlich übel, sie
sieht -- ohne sie als solche zu erkennen -- Bilder der ersten Tage der Pariser
Kommune. 
Kurz lösen sich ihre Krämpfe, dann wird ihr schwarz vor Augen, sie bricht am
Bahnsteig zusammen und wacht erst einige Stunden später in ihrem Bett wieder
auf. 
Bei ihr sitzen ihre Mitbewohnerin und ihre \qq{Partnerin}. 
Sie freut sich über beide nicht, schläft wieder ein.
Das nächste Mal, wenn sie aufwacht, wird Sam neben ihrem Bett stehen und Quinn
weit von Frankfurt wegbringen.\\\\
%
\emph{Szene in vielen kurzen Bildern.}\note{Quinn}
Quinn ist Soziologie Studentin (Nebenfach Kunstgeschichte), ihre Eltern sind
geschieden, der Vater neu verheiratet. 
Nachdem der neue Freund der Mutter Quinn geschlagen hatte, war sie ausgezogen,
nach Frankfurt, in die Stadt. 
Wenig Kontakt zur Mutter -- sie hatte sie nicht schützen können -- öfter beim
Vater zu Hause, um sich um ihren behinderten Halbbruder zu kümmern. 
Vor beiden aber, Vater und Mutter, verschweigt sie, dass sie statt ihre
vorgesehenen Kurse zu besuchen, als Gasthörerin zur Kunsthochschule nach
Offenbach geht, vor allem aber Computer spielt, die neuste Avantgardemusik
verfolgt und das meiste Geld, das sie vom Staat bekommt, weil
die Eltern ihr nichts geben können (oder besser: \emph{wollen}), für Konzerte,
Feiern und Drogen ausgibt. 
So recht zwar jede Studentin damit hat, die Uni zu meiden und sich anderen Dingen
hinzugeben, so sehr ist all dies bei Quinn doch vor allem Ablenkung, Ablenkung
davon, dass die allermeisten Stunden des Lebens eine hässliche Qual sind. 
Verfolgt in jedem Augenblick von den Szenen der Kindheit: der Stimme des
Stiefvaters, den Augen der Mutter, den Schreien des Halbbruders, dem Selbsthass,
bei ihrem behinderten Bruder nur an seine Schreie zu denken und dem eigenen
schlechten Gewissen, alles immer einfach weiterlaufen zu lassen, nichts zu tun,
um sich und andere (wenigstens ein Stück weit) zu befreien. 
Über die anderen, ihre Partnerin, Mitbewohnerin und einige enge Freundinnen
aus der Politgruppe später mehr. 
Der einzige Moment an dem sie sich ganz wohl und ganz bei sich fühlt (von dem
ein oder anderen Mal, wo sie wirklich guten Sex hat, oder Drogen erinnern ans
allererste Mal berauscht-sein abgesehen), ist in den kurzen Minuten vorm
Einschlafen:  
Sie hält einen Joint in der Hand während sie masturbiert, zieht nach dem Orgasmus
noch einmal daran, macht das Licht aus und versucht ihren Körper keinen
Millimeter mehr zu bewegen, während sie in ihrem Bett auf dem Rücken liegt und
Klaviermusik läuft. \emph{Lieder ohne Worte.} 
Erst wenn sie es gar nicht mehr aushält, dreht sie sich zur Seite und kann dann
meist stundenlang nicht einschlafen, bis sie morgens schweißgebadet aufwacht. 
Aber in diesen 15-20 Minuten, in denen Quinn ganz stillliegt, ziehen die
berauschendsten, echtesten und fantastischsten Bilder an ihrem inneren Auge
vorbei, die M sonst nur im Kino zu sehen bekommt. \dots\\\\
%
\emph{Auslaufende Beschreibung der Umgebund, der Gebäude etc.~wie in einem\note{Die Assoziation}
  kitschigen Reiseführer für deutsche Touristen in den italienischen
Berggebieten.} 
Genau diese Momente -- Levitationen, wie Quinn sie insgeheim nennt -- waren
es nun, durch welche die Assoziation, zu der auch Sam gehört,
ein besonderes Interesse an Quinn entwickelte. 
Erst später würde Quinn herausfinden, dass noch eine andere Assoziation aus eben
demselben Grund auf Quinn aufmerksam geworden war. 
Die Ursache für Quinns oben erwähnten Zusammenbruch war indessen, dass beide
Assoziationen zum gleichen Zeitpunkt ihre Bemühungen in die Untersuchung Quinns
gestreckt hatten. 
Etwas davon musste auch Sams Assoziation aufgefallen sein und darum fassten sie
den Entschluss, Quinn schneller als geplant zu sich zu holen. 
Quinn findet sich also, nachdem die Ereignisse seit ihres Zusammenbruchs am
Kirchplatz ihr nur noch momenthaft und neblig in Erinnerung geblieben sind, auf
eine Art altem Kloster, irgendwo in den Bergen wieder. 
Während die Leser*in Quinns Ankommen zunächst harmonisch und erstaunlich
unkompliziert erlebt, offenbaren erst Einblicke in Quinns Träume wie verstörend
der plötzliche Umbruch sich wirklich bei ihr niederschlägt. 
M beobachtet sie zunächst bei einem Spaziergang, wo sie auf Dino trifft. 
Dino konsumiert etwas abseits versteckt eine ihre unbekannte Opioid-ähnliche
Droge. 
Dino reicht Quinn die Pfeife und die Leser*in wird das erste Mal mit der Existenz
der \emph{Arbeitslager} konfrontiert.\note{Arbeitslager}
Nachdem Quinn Dino die Pfeife zurückgibt, sagt Dino, Quinn könne soviel haben,
wie sie mag, es müsse nur ihr >>kleines Geheimnis<< bleiben. 
Mit jeden Drogenrausch erfährt die Leser*in mehr über die Arbeitslager und so
mehr Quinn sich in die Assoziation einlebt, desto mehr steigt ihr Drogenkonsum
an. 
Dino wird neben Sam Quinns erste Bezugsperson.\\\\ 
% 
\emph{Szene bestimmt von inneren Monologen Fs.}\note{Frankfurt: F und S (Streit)}
\emph{Die ganze Szene ist von Wärme und Emotionen geprägt.}
Die Szene wechselt zurück nach Frankfurt, wo Quinns (alte) Mitbewohnerin, F,
Quinns Partnerin S besucht. 
F ist freiberufliche Mediengestalterin, verlässt selten das Haus, aber war
Quinns engste Freundin.
Eine der wenigen Personen denen gegenüber Quinn sich ganz verletzlich zeigen
konnte, was die Leser*in in einem von Quinns Träumen erfährt, in dem Quinn F 
ihr Leiden nach der Ankunft in der Assoziation klagt. 
Als F bei S ankommt, sitzt diese mit Fabian im Garten. 
Beide rauchen und lesen aus einem Buch vor.
>>Weißt du, wo Quinn ist?<<, fragt F. 
Bevor S antworten kann, frag Fabian: >>Wer ist Quinn?<< 
F stürmt aus dem Garten, völlig zerstört vom plötzlichen
Verschwinden Quinns und erbost über die schlechte Kommunikation in polygamen
Beziehungen.\\\\
% 
Quinn hat ihre erste \qq{Session} mit Sam.\note{Session I: Zigaretten}
Sie soll levitieren. 
Dabei gibt Quinn ihr Anweisungen. 
Quinn soll sich ans Frühstück erinnern. 
>>Hast du nach dem Kaffee eine Zigarette geraucht?<< 
>>Nein.<< 
>>Tu es.<< 
Als Quinn wieder aufwacht, zählt sie die Zigaretten in ihrer Schachtel. 
Eine fehlt. \emph{(Szene Pariser Kommune)}\\\\
% 
\emph{Die ganze Szene ist von Wärme und Emotionen geprägt.}\note{Frankfurt: Entschuldigung (Versöhnung)}
S besucht F und entschuldigt sich.
Sie reden lange über Quinn, ihr Verschwinden, ihre jeweiligen Beziehungen mit
ihr. 
Beide genießen das Gespräch, obwohl sie früher nie besonders viel miteinander
anzufangen wussten.\\\\
%
\emph{Szene geschrieben, wie ein Theaterstück.}\note{Die Tafel}
Sam, Dino und Quinn sitzen mit ein paar anderen aus der Assoziation nach dem
Essen noch länger am Tisch (eine große hölzerne Tafel) und unterhalten sich. 
Hier erfährt zumindest die Leser*in das erste Mal einiges über die Assoziation: 
ihren Namen, ihre Ziele, etwas übers Levitieren. 
Die Assoziation weiß noch nichts von der anderen, feindlichen Assoziation. 
Das Gespräch endet damit, dass Quinn fragt, wieso sie denn zusammengebrochen
sei, wenn sie sie schon so oft \qq{untersucht} hätten.\\\\
%
\emph{Erneut eher distanziert, Reiseführer-mäßig beschrieben.}\note{Rausch}
Quinn und Dino rauchen vorm Schlafengehen eine Pfeife. 
Dino versucht an das Gespräch am Küchentisch (warum ist Quinn ohnmächtig
geworden?) anzuknüpfen, doch Quinn wert ab (>>lass uns einfach zu zweit die Ruhe
genießen, ja?<<).
Szenen aus dem Arbeitslager.\note{Arbeitslager}\\\\
% 
Hier sollten wir einiges über die aktuelle Welt und die Assoziation\note{Zusammenfassung}
zusammenfassen. 
Der Spätkapitalismus ist nach einer globalen Pandemie im Jahre 2025 in eine
schwere Akkumulationskrise geraten. 
Als Folge verschärft sich die internationale Blockbildung. 
Als schließlich aufgrund der fortschreitenden Klimakrise und der am Boden
liegenden Wirtschaft quasi auf einen Schlag die Zentren des globalen Südens
zusammenbrachen, sah der Westen darin seine neue Chance. 
Die globalen Player der Nato-Staaten -- allen voran Deutschland, Frankreich und
die USA -- setzten in sämtlichen Ländern Südamerikas und Afrikas
Marionetten-Regierungen ein, stabilisierten zunächst durch massive Kredite die
dortige Wirtschaft, hoben aber zugleich sämtliche sozialstaatlichen und
arbeitsrechtlichen Programme und Regulierungen auf, um zunächst Anreize für
enorme privatwirtschaftliche Investitionen zu schaffen, aber schließlich, Stück
für Stück jene Lände in moderne Arbeitslager zu verwandeln. 
Die dort produzierten Waren konnten, neben den kapitalistischen Zentren des
Westens, in den durch die schlagartig steigende Produktivität wirtschaftlich
vernichteten Ost-Block abgesetzt werden. 
Der Ost-Block (d.h.~vor allem China, Russland, Iran) wurde dadurch nicht nur
indirekt militärisch ausgeschaltet, sondern auch vom Westen vollständig abhängig
gemacht, aber zwecks Absatzmärkten noch gerade so am Leben gehalten.
Nicht so Südamerika und Afrika, die fortan \emph{Industrial Zone 1} und
\emph{Industrial Zone 2} (IZ-1, IZ-2) genannt wurden: 
Hier herrschte vollkommene militärische Kontrolle und Degradierung sämtlicher
Menschen zu modernen Arbeitssklav*innen, die nicht Teil des Kontroll- und
Überwachungsapparats waren.
Zunächst herrschte auch in den kapitalistischen Zentren ein offenes Klima der
Angst und Repression, zur Schau gestellter Autorität und Menschen wurden für die
kleinsten Vergehen in die neuen Arbeitslager abgeschoben. 
Auch fand ein >>ethnic cleansing<< statt: 
Während \qq{normale} Verbrecher in IZ-1 verfrachtet wurden, transportierte M
sämtliche Menschen, die bis in die 5.~Generation eine Fluchtgeschichte aus den
nicht westlichen Ländern aufwies, in IZ-2. 
Doch mit zunehmender Prosperität der Zentren, wurden liberalere Gesetze
erlassen, eine gewisse ethnische Pluralität forciert, nicht der
binären-heterosexuellen Norm entsprechende Lebensentwürfe legalisiert und vor
allem die offene, autoritäre Rhetorik und Praxis zurückgefahren, so weit, dass
das Leben in IZ-1 und -2 idealisiert wurde und die Menschen in den Zentren
\qq{guten Gewissens} ihr Leben leben konnten. 
Obwohl der Lebensstandard stieg, die Effekte des Klimawandels durch moderne
Technologie weitgehend im Zaum gehalten werden konnten und alternative
Lebensformen ihren Raum bekamen, war auch hier kapitalistische Ausbeutung mit
Nichten überwunden. 
Es ließe sich sogar sagen, dass in dem Maße, wie sich diese Entwicklung -- für
die Menschen in den Zentren (!) -- \qq{zum Guten} wandelte, dass was M
\qq{entfremdete Arbeit} oder \qq{bürgerliche Kälte} nennt, zunahm. 

Dass überhaupt eine so umfassende ideologische Verblendung Fuß fassen konnte,
hatte allerdings noch einen zweiten Grund: 
Ein gewisses -- streng geheimes -- Militärprogramm entwickelte eine Technik --
ähnlich des luziden Träumens -- durch die für kurze Augenblicke die
Raum-Zeit-Dimension nach sämtlichen Achsen durchschritten werden konnte.
Diese Technik wurde vermehrt eingesetzt, um die Erinnerungen an die Gräuel nicht
nur in IZ-1 und -2, sondern auch innerhalb der Zentren, aus dem (kollektiven)
Bewusstsein der Menschen zu verdrängen. 
Trotz der enormen Macht, die von dieser Technik ausging, hatte sie ihre Grenzen
und wurde nicht über die ideologische Manipulation hinaus eingesetzt -- und auch
hier nur in bestimmten Fällen. Denn, wie beim luziden Träumen auch: sobald M zu
sehr versucht den Traum zu beeinflussen, zu sehr versucht seinen \emph{Willen}
ins Spiel zu bringen, löst er sich auf. 
Diese streng geheime militärische Assoziation ist diejenigen, von der \qq{unsere}
Assoziation noch nichts weiß. 
Aber selbst diese so geheimgehaltene Technik musste doch auf unbekanntem Wege 
irgendwie durchgesickert sein, denn einige wenige außerhalb des Militärs, die
von ihr erfuhren und sie schließlich selbst beherrschten gründeten eben jene
Assoziation, in die Quinn nun aufgenommen wurde. 
Ihr Ziel: durch das Levitieren -- wie sie die Technik nannten -- die Etablierung
der Arbeitslager zu verhindern, indem sie mikroskopische Eingriffe in die
Geschichte machten. 
Da diese -- aus oben genanntem Grund -- so winzig sein mussten, dass sich die
Levitation nicht auflöste, erreichten die Eingriffe vor allem Liberalisierung,
Verhinderung von Blockbildung und Kriegen und Zurückdrängung des westlichen wie
östlichen Imperialismus.
Doch erst viel später sollte sich herausstellen, dass nur diese Änderungen
(welche die Assoziation, wie eben beschrieben, ja tatsächlich erreichen konnten) 
die tiefe Akkumulationskrise hervorbrachten, da ein globaler, egalitärer
Kapitalismus, frei von Überausbeutung nichts mehr hat, was er akkumulieren kann. 
Als Konsequenz wurden die Arbeitslager gegründet.\\\\
%
\emph{Die Szene zwischen F und S ist ein letztes Mal von Wärme und Emotionen}\note{Frankfurt: Betrogen (Nähe oder Konflikt)}
\emph{geprägt.}
S besucht erneut F. 
Statt in der Küche, gehen sie durch Quinns Zimmer, betrachten Fotos,
Gegenstände. 
S erinnert Quinns Geruch. 
F erzählt S von Quinns Vergangenheit: Vater, Mutter, dass sie geschlagen wurde. 
S erzählt schließlich, dass Quinn sie einmal betrogen hat: >>Das wusste ich
alles gar nicht. Oder kaum etwas davon. Wusstest du, dass Quinn mich einmal
betrogen hat? Sie erzählt sonst so wenig von \dots{} sich.<< 
Das nächste Mal, dass F und S sich treffen, werden sie vor allem schweigen. 
S wird denken, es sei ihre Schuld, weil sie vom Betrügen erzählt hat. 
F hängt Gedanken an ihren Vater nach, der am Tag zuvor in IZ-2 abgeschoben
wurde, zur \qq{Alkoholtherapie}. 
Erst im Treffen darauf wird F das erzählen. 
Sie unterhalten sich, dann über IZ-1 und -2. 
Ob Quinns Verschwinden etwas damit zu tun hat?\\\\
%
Tagebucheintrag von Quinn über alltägliches: Essen, Trinken, Kacken, ihre
Tage.\note{Tagebuch I}\\\\
% 
Es ist Nachmittag.\note{Session II: Die Männer}
Quinn hat erneut eine Session mit Sam. 
Sie soll sich mit einer Person aus der Assoziation unterhalten. 
Das Gespräch soll wieder nicht zu weit entfernt liegen, am selben Morgen, nach
dem Frühstück. 
Sam führt Quinn in die Levitation ein, dann redet Quinn mit Dino, fragt, ob
Dino ihr auch mal was mitgeben könnte, wenn sie Dino gerade nicht finden könnte,
oder es zu spät sei.
\emph{Die Frage nach den Drogen reflektiert auf das Verdrängen des Betrügens.}
Dino sagt nach kurzem Zögern zu, aber Quinn müsse vorsichtig sein, es müsse
wieder >>ihr kleines Geheimnis bleiben<<. (Quinn fragt sich, ob Dino das sagt,
um aufregend zu wirken, wie in einem Film.) Als Quinn gerade die Levitation
beenden will, sieht sie zwei schwarz gekleidete Menschen etwas im Hintergrund
hinter Dino stehen. 
Sie wirken unpassend, wie Fremdkörper. 
Nach dem Aufwachen erzählt sie Sam von den Menschen. 
Sam will wissen, mit wem Quinn gesprochen hat und worüber. 
Quinn verweigert zu antworten. 
Sie streiten sich. 
Quinn wird ihre nächste Levitation mit jemand anderen haben
(\qq{Cyber}-Security). >>Bitte Sam, tu mir das nicht an. Ich weiß nicht ob ich
es mit jemand anderem könnte.<< >>Es ist deine Entscheidung.<< 
\emph{Szenen Pariser Kommune.}\\\\
%
\emph{Obwohl Dino nicht dabei ist, erneut distanziert.}\note{Alleine}
\emph{Wie aus einem Reiseführer wird die Landschaft beschrieben. 
>>Was ein feiner Ort um alleine zu sein. Hier können Sie sich erholen.<<}
Quinn raucht alleine eine Pfeife.\note{Arbeitslager}\\\\
%
Quinn träumt.\note{Traum: Gespräch mit F.}\\\\
% 
\emph{Die Szene ist nicht mehr in der üblichen Wärme geschrieben,}\note{Frankfurt: Stille (Streit)}
\emph{sondern kälter, irgendwie distanzierter.} 
Treffen S und F. Sie sitzen wieder in Quinns altem Zimmer, doch wissen sich
nichts zu sagen, obwohl sie die gleichen Sachen riechen, d.h.~Quinns Geruch an
ihren Möbeln und Kleidern.
Aber ganz Anderes geht ihnen durch den Kopf. (F: Mein Vater wurde abgeschoben
nach IZ-2, zur \qq{Alkoholtherapie}. S: Ist F mir böse, weil ich ihr von Quinns
Fremdgehen erzählt habe?)\\\\
% 
Session mit Vox.\note{Session III: Vox}
Größere räumliche Distanz und weiter zurück in die Vergangenheit. 
Eine Tablette aus dem Blister eines Künstlers entfernen. 
Quinn soll ganz genau auf die Umgebung achten. 
Quinn gelingt zunächst das Levitieren nicht: 
Gerade will sich der Traum zusammensetzen, Vox spricht die ersten Wörter, um den
Traumgehalt zu initiieren, da erscheint das Gesicht von Vox vor Quinns innerem
Auge und der Traum zerbricht. 
Vox (von Anfang an Quinns Fähigkeiten infrage stellend) wird zunehmend
ungeduldig (was es Quinn nicht leichter macht sich in den Traum zu versenken)
und schließlich sogar wütend: >>Ich habe mir schon gedacht, dass größere Sprünge
dir schwerfallen würden!<< 
>>Vielleicht würde es mir helfen, wenn du mir etwas mehr Zeit gibst
reinzukommen.<< 
>>Es gibt noch eine andere Technik<<, sagt Vox und legt sich neben Quinn auf den
Boden, beginnt sie zärtlichen an den Unterarmen zu streicheln. 
(Die Szene könnte als Vergewaltigung interpretiert werden.)
Vox fängt langsam wieder an zu sprechen. 
Quinn ist die Berührung und die Nähe unangenehm, dennoch bricht langsam der
Traum über sie hinein. 
Quinn läuft einen langen Gang entlang. 
Gefängniszellen links und recht. 
Eine Tür steht offen. 
Quinn tritt herein. 
Neben einem kargen Feldbett liegt auf einem kleinen, an der Wand befestigten
Metalltisch der Blister. 
Quinn nähert sich ihm. 
Kurz bevor sie ihn greifen will, dreht sie sich um und sieht zwei schwarz
gekleidete Gestalten. 
Eine von ihnen nähert sich Quinn. 
Die Situation wird bedrohlich. 
>>Stirbst du hier, stirbst du auch dort, überall<<, tönt es tausendfach hallend 
durch Quinns Kopf. 
Wie in Schockstarre blickt Quinn seit ihrem Eintreten der Person in die Augen,
die sich noch nicht wegbewegt hat von der Tür. 
Die Augen sind tiefblau, mit einem kleinen grün-schimmerndem Punkt darin. 
Die andere Wache kommt Quinn gefährlich nahe, zückt mit einer plötzlichen
Bewegung einen Draht und hält ihn mit beiden Händen, wie um ihr jeden Augenblick
die Kehle zuzuschnüren, als sich mit einem Mal die Szenerie wandelt. 
Quinn wird durch die Augen der zweiten Person eingezogen, tränkt sich mit ihr in
Kindheitserinnerung, wo sie erkennt, dass sie zurückkommen kann\note{1.~Kindheitserinnerungen}
hinter die erste Wache, um so der Situation zu entkommen. 
Die mäßig sexuelle Rückblende in eine alternative Kindheit mit der zweiten Wache
steht symbolisch für die Vergewaltigung von Vox. 

Quinn fährt erschrocken aus ihrer Levitation hoch. Schwitzend blickt sie sich
noch halb liegend, halb sich nach oben stützend um. 
Vox ist bereits aufgestanden, blickt sie besorgt an. 
>>Was ist passiert?<< 
>>Ich dachte, du kannst alles sehen, was passiert, anders als Sam.<< 
>>Das kann ich auch, jedenfalls alles was da ist. Aber ...<<
Quinn blickt Vox fragend an. 
>>Plötzlich stand alles. still.<< 
Quinns Blick wird mit einem Mal leer, ihre Gedanken fangen an zu rasen als sie
bemerkt, dass Vox nicht sehen konnte, was passiert war, nachdem sie in die Augen
des Wächters getaucht war. 
Quinn ist gerührt, dass Vox sich anscheinend Sorgen gemacht hatte, aber immer
noch sauer, dass Vox sich einfach neben sie gelegt hatte, und für Vox Art
überhaupt.
Quinn fühlt sich seltsam unwohl, etwas passte nicht mit Vox, eine merkwürdige
Kälte ging von Vox aus. 
Quinns Blick bleibt starr und leer. 
Ohne Vox anzuschauen, steht sie auf, nimmt ihren Beutel und dreht sich zur Tür,
um den Raum zu verlassen. 
>>Vielleicht war ich zu hart zu Beginn unserer Stunde. 
Das tut mir Leid. 
Aber sag mir: was ist da passiert?<<
Quinn hält eine Sekunde inne, geht dann aber doch weiter durch die offen
stehende Tür hinaus und auf ihr Zimmer. 
Vox bleibt gleichfalls wie im Schock stehen; geschockt von Quinns\note{In Form von Brief Vox an Zaro.}
Blick und plötzlicher Stille. \emph{(Szenen Pariser Kommune)}\\\\
\subsubsection{\textsc{TEIL II -- Der Zyklus}}
%
\begin{center}
  \qdots{}\\
\end{center}
\bigskip
%
\emph{(Erster) innerer Monolog Quinn.}\note{Sie spricht.}
Quinn wacht auf und raucht eine Pfeife. 
Sie ist müde, ihre Augen gehen kaum auf, sie wird das Gefühl nicht los, aus
einem nie endenden Strudel entlassen worden zu sein. 
Ihr Kopf brennt.
Trozdem es schlimmer geworden war die letzen Zeit, nach dem Aufstehen als erstes
zu rauchen war neu. 
Einstürzende Neubauten. 
Sie reflektiert über die erste Session mit Vox, als wäre keine Zeit vergangen. 
Hunger, aber keine Lust (die anderen zu sehen). 
Es werden keine Bilder vom Arbeitslager gezeigt, stattdessen reflektiert Quinn
selbst über >>komische Bilder<<, >>Gewalt und Ausbeutung<<, die ihr inneres Auge
plagen. 
Quinn beschreibt diese Bilder mit Ekel, Angst, Erschrecken, Trauer, aber ohne,
dass es von Quinn selbst ausgesprochen wird, ergeben sie sich deutlich auch als
Orte der Lust.
Es vergeht mehr Zeit. 
Der Gedanke, M könne nach ihr suchen, huscht vorbei. 
Quinn fängt an darüber zu sinnieren, mit Sam zu reden:
Zunächst nochmal über Vox, dass Vox eine der wenigen Menschen >>hier<< ist, bei
dem sie sich noch unwohl fühlt.
Dann kommt der Wunsch hoch, einer Ahnung von Wohl-fühlen nachzugehen. 
Dort drüber, hinter den Augen, fühlt es sich warm an. 
Träume deuten Schmerzen von Verlust an. 
>>Sam davon erzählen?<<
Quinn beginnt ein Gespräch mit Sam darüber, den Wächtern nahe zu sein, zu
fantasieren.
Beim Reden spürt sie langsam, dass da noch eine zweite Emotion, eine Schuld
ist. 
Schuld, sie -- die Wächter -- verlassen zu haben, ohne sich zu verabschieden.
Einfach verschwunden. 
Für die Leser*in ist zunächst nicht klar, ob dieses Gespräch mit Sam tatsächlich
stattfindet. 
Bis Dino (oder jm.~anders?) Quinn plötzlich an der Schulter packt. 
Quinn wacht aus ihren Gedanken auf, steht vor Sam, neben ihr, verschwitzt, Dino.
Dino hätte sie überall gesucht, auf Befehl von Vox, Quinn müsse zu ihrer Session
kommen. 
Sam steht fragend da, als hätte Sam noch etwas von Quinn erwartet, als hätte
Quinn die ganze Zeit über kein Ton herausgebracht. 
Dino begleitet Quinn zur Session mit Vox. 
Sam bleibt mit offenem Mund, besorgt stehen, wie Vox, damals nach der ersten
Session.\\\\
%
Mitten in der Levitation reißt Vox Quinn heraus.\note{Eroberung}
Es fühlt sich an wie wenn M beim Schwimmen plötzlich an den Beinen gepackt wird:
Der Körper, besser: die Gedanken wollen weiter, doch die Welt steht
still.\note{Abbruch mit Szenen Pariser Kommune kontrastiert.}
>>Was ist?! Warum hast du das gemacht?<< 
>>Sei still!<< 
Draußen hört Quinn Schreie. 
Einen Schuss, zerberstendes Glas. 
Aufruhr.
Sie macht einen Schritt auf die Tür zu, Vox versucht sie festzuhalten. 
Quinn reist sich los. 
Ihr Kopf brennt. 
Sie ist verkatert, fühlt sich zunehmend ausgelaugt. 
Quinn fällt durch die Tür und landet auf allen vieren. 
Zwei Wächter stehen auf dem Platz, hinter ihnen die endlosen Berge. 
Sam steht schützend vor zwei Kindern, Dino liegt auf dem Boden. 
Aus dem Rohr der Pistole dampft es, Scherben fallen noch immer aus dem großen
Fenster des Gemeinschaftsaals, wie Blut, das aus einer Wunde fließt. 
>>Wo ist Zaro?<<, fragt einer der Wächter und richtet die Pistole auf Dino,
blickt dabei Sam an. 
Dass Quinn \qq{aufgetreten} ist, scheint ihnen gar nicht aufgefallen zu sein. 
>>Hey!<<, brüllt Quinn. 
Versucht sich aufzurichten, doch bricht wieder zusammen, ihr Knie schmerzt noch
zu sehr vom Sturz. \emph{Quinn soll hier schwach, abgerockt wirken. 
Die Illusion einer noch ungenutzten Macht Quinns, soll grausam zerplatzen.} 
Der Wächter ohne Pistole dreht sich zu Quinn. 
Sie blick ihm in die Augen. 
\emph{Kindheitserinnerung.}\note{Kindheitserinnerungen}
Doch etwas ist anders. 
Die Szene ist von etwas Unbekanntem bestimmt, anderes Licht, Quinn fühlt sich
weniger sicher >>in seinen Augen<<. 
Dennoch lässt sie sich zeigen, wo sie hin muss, um der Gefahr zu entkommen. 
Überrascht löst sie sich, fällt. 
Quinn macht ihre Augen wieder auf und steht vor einem der Wächter. 
Er packt sie: >>Ach, du bist wirklich hier? 
Das hätten wir ja nicht gedacht. 
Dann nehmen wir halt dich mit.
Komm<<, sagt der Wächter, das letzte Wort an den anderen schwarzen Mann
gerichtet. 
Dann zu den andren: >>Auf Wiedersehen.<<
Quinn sieht nur noch in Sams Blick den Ausdruck von Erleichterung, dass
Schlimmeres nicht geschehen ist. 
Das schmerzt. 
Dann fallen ihr die Augen zu. 
>>Willkommen zurück.<< 
So wacht sie bei den Wächtern auf. 
Wird von ihnen versorgt. 
Kalter Entzug. 
Schweißgebadet. 
Träume. 
Rückkehrgefühl, Wut und \emph{Schuld}. 

\subsubsection{\textsc{TEIL III -- Illusionen}}

\emph{Langsam erfährt sie hier von den Arbeitslagern und wie sie entstehen konnten und
von der Rolle der Assoziation in der ganzen Sachen. 
Während die Wächter zwar versuchen sie wieder auf die Beine zu stellen, bringt
der kalte Entzug sie langsam um. 
Schließlich sieht sie noch einmal Sam und Dino. 
Tränen, als ihre Bilder langsam verblassen.}\\\\
%
Kube und Janes gedämpften Stimmen dringen zu Quinn herüber, die\note{Kube \& Jane 1: ...}
entweder auf einem Krankenbett oder in einem großen Kissenbefülltem Bett unter
einem Baldachin liegt.
Die Szene beginnt bei Kube und Jane und offenbart erst später Quinn als
Zuhörerin. 
Das Gespräch von Kube und Jane mimt die zahllosen Gespräche von F.~und S.;
Auch sie werden sich wiederholen und sich durch den dritten und letzten Teil
ziehen, die Szenen aus Frankfurt ersetzend, nur diesmal beobachtet vom
halbbewussten Zustands Quinns, zunächst nach ihrer Bewusstlosigkeit, dann im
Delirium ihres Entzugs.
>>Sind wir uns sicher, dass sie es ist?<<, fragt Kube, >>ist sie wirklich
zurück?<< 
>>Die Koordinaten könnten passen, wir haben zurückverfolgt, wo die
Verschiebungen herkamen, wenn unsere Berechnungen~--<< 
>>Es fühlt sich auch so an, als wäre sie es. Wieder da. Ich habe mir ihre Haare
anders vorgestellt, kürzer\dots{}<< 
Es folgt die erste ausführlichere Beschreibung von Quinns Äußerem: 
Schulterlange Haare, ungekämmt, unsauber geschnitten, rundes Gesicht mit
Sommersprossen, ein runder Mund, wenn sie lacht -- im Traum z.B.~-- spitz, leicht
spöttisch, wenn er geschlossen ist, schmale Schultern, dünne Arme. 
Lange, schöne Finger, blaue Augen mit leichten Lachfalten und großen
Augenringen, tiefblau-lila, kleine, spitze, schöne Nase. 
Nicht im \qq{klassischen} Sinne, aber gerade die Sympathie, die Haut ihres
Gesichts, das unperfekte ihres ganzen Ausdrucks und Auftretens, hat etwas
wahnsinnig anziehendes, dass M hübsch nennen könnte, dass für einen Menschen,
der sie liebt, das schönste Wesen auf Erben sein könnte. 
Ihres Brüste sind flach, ihr Bauch, schlank aber nicht dünn. 
Durchschnittliche Beine, und wahnsinnig feine Füße. 
Kubes Sinnieren über Kubes Diskrepanz zwischen Erwartung und Wirklichkeit wird
immer wieder unterbrochen von Janes Rekonstruktionen der Levitationspfade, die
Sicherheit geben sollen, dass der Mensch, der jetzt hier liegt, der gleiche ist,
mit dem bis vor ungefähr einem Jahr und für ca.~drei Jahre Kontakt aufgebaut
wurde.
Die Leserin erfährt hier einerseits neue Details übers Levitieren, die bis
hierher verborgen geblieben waren: Wie die Kontaktaufnahme teils \qq{Bilderlos}
verläuft, dass das dadurch nicht klar sein konnte, ob Quinn, tatsächlich Quinn
ist, dass Levitationen Spuren hinterlassen\qdots{}, und andererseits, ergibt
sich schließlich für sowohl Jane als auch Kube, dass die Person dort auf dem
Bett auf jeden Fall Quinn sein muss. 
Jane durch die Rekapitulation der Berechnungen, Kube durchs Nachempfinden. 
>>Wusstest du, dass ich von uns allen zuerst mit ihr Kontakt aufgenommen und
\emph{begleitet} habe?<< 
>>Natürlich.<<
Die Leserin erfährt hier auch, das erste mal von der ganz anderen emotionalen
Seite des Levitierens: Um eine Person zu \qq{testen} reißt M an einen
RaumzeitPunkt und begleitet ihre (zufällige) Levitation. 
Dabei speist sich die Szene aber bloß aus den eigenen Phantasien, nur dasjenige,
auf das M sich konzentriert, nur ein minimaler Anteil kann bewusst gesetzt
werden. 
Quinns erste Levitation, \emph{cigarettes}, die sich noch in den Räumen der
Assoziation abgespielt und nur wenige Stunden in der Vergangenheit gelegen
hatten, waren deshalb noch ganz Lebensecht, \emph{original}\footnote{
  Als \emph{Original} wurden im Sprech der Levitierenden der Anteil der
  Levitation bezeichnet, der nicht fantasiert ist, sondern eins zu eins der
  Realität zum entsprechenden Raumzeitpunkt gleicht. 
  Im Extremfall ist selbst alles unmittelbar mit dem Ziel der Levitation
  verknüpfte bloß metaphorisch dargestellt, speißt sich aus dem Unbewussten der
  Levitierenden: Ein gesuchter Schlüssel findet sich durch ein Gespräch, der
  Versuch, eine Begegnung zwischen zwei Personen darzustellen, gelingt durch das
  Graben eines Kanals zwischen zwei Seitenarmen eines Baches, die Information zu
  einem geschriebenen Brief liegt in der Anordnung der Tannen eines Mischwaldes
  und ihres zahlenmäßigen Verhältnisses zu den Laubbäumen\dots{} Ansonsten
  werden immer mehr Originale zu dem Grad durch Metapher und Fantasie ersetzt,
  wie der Raum unbekannter ist, die Zeit weiter in der Vergangenheit liegt (der
  Abstand von Ich zu Raumzeitpunkt bestimmt das Verhältnis von Original zu
  Fantasie).
}, während das ganze Gefängnis bei der dritten Levitation nichts mit der
Wirklichkeit zu tun hatte, sondern bloß das Gefühl des Eingesperrt-Seins durch
die Assoziation widerspiegelt. 
Levitieren setzt dadurch aber eine extrem hohe Freilassung libidinöser Energien
voraus. 
Werden Menschen z.B.~getestet, geht es gerade darum, wie stark diese sich auf
die unbewusst Präsenz einlassen können\footnote{
  D.h.~wenn Menschen gemeinsam levitieren (entweder bewusst, oder indem sich
  eine Person in die Levitation einer anderen einschleicht -- wie z.B.~die Wächter
  in die Quinns), setzt sich der durch Metapher und Fantasie gebildete Anteil
  aus den unbewussten Impulsen beider, oder \emph{aller} Beteiligten zusammen. 
  Je mehr sich die Fantasien aufeinander einlassen, können sie gemeinsam
  gestalten, andererseits entstehen Risse, die Levitation wird weniger klar,
  unabgeschlossene, abbrechende Szenen nehmen überhand, die Levitation erscheint
  eher wie Erinnerungsfetzen eines Traumes, statt wie ein beinahe filmartiges
  Miterleben.
  Dass Quinn damals die Wächter konnte \emph{entdecken} konnte, lang nur an
  einer -- für eine \qq{Neue} ausgesprochen hohe -- Freilassung libidinöser
  Energie.
} und ihrerseits libidinöse Energiefreilassung. 
Alle Besuche (der Assoziation und der Wächter) waren Besuche bei Quinns
unbewussten Levitationen -- ihre glücklichsten Momente, kurz vorm Einschlafen. 
(Es werden überhaupt nur Menschen getestet, die selbst beginnen zu levitieren --
unbewusst -- und dadurch schon eine gewisse Affinität zeigen. 
Synchronisieren sich dabei die unbewussten Fantasien der Beteiligten, ist die
Wahrscheinlichkeit hoch, dass der getesteten Person auch bewusstes Levitieren
gelingt.)
