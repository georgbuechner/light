Nicht alle Charaktere sind (von Anfang an) Teil einer der folgenden Parteien.
Die >>Parteien<< bilden eher etwas wie rivalisierende Weltanschauungen mit
entgegenstehenden Zielen. Die meisten Parteien streben allerdings an, die
>>normalen<< Spieler*innen von ihrer Position zu überzeugen.
\begin{itemize}
  \item[] Die Parteien (auch die Zeitwächter*innen) leben in allen drei Zeiten
    und sie alle >>normaler<< Teil des Dorfes.
\end{itemize}
Die Zugehörigen zu P-1 und P-2 (zu Beginn jeweils 2-3 Spieler:innen), wissen als
(vermutlich) einzige, dass Zeitreisen möglich ist. Sie besitzen außerdem jeweils
eine Zeitreisemachine, die aber nur noch >>Treibstoff<< für jeweils 1-2 Zeitreisen
haben. Durch das Lösen verschiedener Aufgaben (entweder mache X, um an
Treibstoff zu kommen, oder mache Y, um Z zu erreichen, wobei du zufällig auf
Treibstoff stoßen wirst), können die Parteien an neuen Treibstoff für mehr
Zeitreisen kommen.

Die Konflikt(e) und Ziele von P-1 und P-2 sind noch nicht näher erdacht. Sie
sollen aber spannend, innovativ (nicht einfach Gut gegen Böse) und vor allem gut
in die Charakter-Ebene eingeflochten sein (z.B.~mit dem Mord verknüpft).
