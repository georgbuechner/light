\emph{Der Main-Plot dreht sich hauptsächlich um alles was zur \textbf{Plot-Ebene}
gehört, d.h.~der Konflikt zwischen P-1 und P-2. 
Er lässt sich aber nur konzeptionell von den anderen Ebenen (vor allem der
Charakter-Ebene) abtrennen. Im Erlebnis der Spieler*innen sollte er nicht als
ein losgelöstes Element wahrgenommen werden. 
Auch die Spielmechanik-Ebene gehört hierher: der Countdown und die drohende
Katastrophe, die den beiden Gruppe zunächst \emph{nicht} bewusst ist.\\
Der Main-Plot ist bisher noch am wenigsten ausgearbeitet.}\\\\
%
Die Feindschaft dieser beiden Gruppen entzündete sich, wie bereits in
\nameref{sec:setting} angemerkt an dem Mord an Laura Palma im Jahre 1968 und
drück sich zunächst in einem anderen Umgang mit dem Mord aus. Während
\nameref{sec:p1} den Tod Laura Palmas bearbeiten und im Gedächtnis des Dorfes
behalten will, strebt \nameref{sec:p2} an, ihn aus dem kollektiven Bewusstsein
zu verdrängen.
\emph{Wichtig ist, dass zunächst nicht klar ist, was mit Laura Palma passiert
ist: ist sie verschwunden, tot, oder tatsächlich ermordet worden?}

Der Main-Plot sollte aber irgendwie damit verbunden sein, dass die Parteien in
die Zeit zurück an die Gründung des Dorfes gereist sind, um Einfluss auf die
Entwicklung des Dorfes zu nehmen. Laura Palma ist das Kind dieser
Zeitreisenden.

Die dirtte Ebene ist, dass Laura Palma selbst in die Zeit zurückgereist ist und
die >>nihilistischen<< Jugendlichen -- die so oder so einen Mord geplant hatten
-- angeregt hat, Laura Palma (d.h.~die jüngere Version ihrer selbst)
umzubringen, weil Laura Palma der Überzeugung war, dass sie als
Kind-das-nicht-hat-sein-sollen, der Grund für die nicht-endenden Konflikte
ist.\\\\

\emph{To be continued...}
