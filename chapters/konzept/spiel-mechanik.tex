Durch äußerliche Merkmale (z.B. Choker. Siehe:
\nameref{ch:offene_fragen}), wissen die Spieler:innen, wer zu ihrer Zeit
gehört (erklärt in den Spiel-regeln *ohne* dabei auf die Zeiten einzugehen:
>>z.B.~Menschen mit einer anderen Farbe Chokern, siehst, fühlst und hörst du
nicht (-- du kannst sie aber riechen?)<<.\\\\
%
Zunächst sind die Zeiten zusätzlich \emph{räumlich} getrennt (z.B. auf
unterschiedlichen Stockwerken der Unterkunft, die aufgrund von
Unwetterwarnungen verschlossen sind).\\\\
%
Reisen durch die Zeit wird durch einen Wechsel des Chokers dargestellt.\\\\
%
Es ist nur eine bestimmte Zahl an Zeitreisen erlaubt, dann tritt die
Katastrophe ein. Evtl.~gibt es einen physischen Countdown, der auch entdeckt
werden kann (so könnte z.B.~\nameref{sec:p3}, die Partei der
Zeitwächter*innen, zu Beginn dieses physische >>Gerät<< entdecken, ohne
genau zu wissen, womit sie es zu tun haben).\\\\
%
Die Zahl der erlaubten Zeitreisen bis zum Eintritt der Katastrophe
(Siehe: \nameref{ch:offene_fragen}), sind so begrenzt, dass sie gerade so
reichen, die für gewissen Spielziele \emph{notwendigen} Reisen zu
absolvieren.\\\\
%
Stirbt eine Spieler:in während des Spiels, bekommt sie eine neue Rolle
aus \nameref{sec:zc} (2066) zugewiesen. Um zu vermeiden, dass die erste
Person die stirbt zunächst alleine ist, wird die NSC, welche die ermordete
Person spielt, zu Z-C hinzukommen, sobald die erste Spieler:in stirbt.\\\\
%
\qcert{\nameref{sec:zc} ist entweder auf der Seite von \nameref{sec:p1}, oder
auf der Seite von \nameref{sec:p2} (das kann z.B.~dadurch gelöst werden,
dass bestimmte >>Meilensteine<< bzgl.~der Ziele von P-1 und P-2 dazuführen,
dass die nächste Person, die zu Z-D hinzukommt, d.h.~die stirbt, das Ziel
von entweder P-1 oder P-2 unterstützt.)}\\\\
%
\qcert{Option A: Die >>Katastrophe<< ist das Eintreten der \nameref{sec:zc} in die
anderen Zeiten (Problem: aus bildungspolitischen Gründen, d.h.~aufgrund der
Parallelen zur Klimakrise, sollte die Katastrophe abwendbar sein, was das
Eintreten von Z-D optional machen würde). Option B: Z-C kommt zu einem fixen
Zeitpunkt zum Spiel dazu.}\\\\
