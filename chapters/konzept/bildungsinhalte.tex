Die Katastrophe steht symbolisch für den Klimawandel, da sie durch exzessives
Nutzen von >>schädlicher<< Technik (zu viele Zeitreisen) ausgelöst wird. 
Wie auch der Klimawandel muss das Faktum der drohenden Katastrophe zunächst
entdeckt und dann überhaupt an alle (Menschen unterschiedlichster Interessen --
und unterschiedlicher Zeiten) vermittelt werden.\\
Wie auch der Klimawandel, muss die Katastrophe durch gemeinschaftliches Handeln,
gegen die unmittelbaren persönlichen Interessen, durchgesetzt werden.
(Spielmechanik-Ebene)\\\\
%
Die drei Zeiten sollen unterschiedliche Stufen der feministischen Bewegung
thematisieren (first, second, third wave feminism). In \nameref{sec:zc}, ist der
Feminismus in der Auflösung der Geschlechts-Identitäten aufgehoben.
(Charakter-Ebene)\\\\
%
Durch das Zusammenführen der Zeiten geraten (auch) die divergierenden
politischen Ansichten in Konflikt. (Zeit-Ebene)\\\\
%
\qcert{Der Kampf zwischen \nameref{sec:p1} und \nameref{sec:p2} thematisiert
verschiedene (psychoanalytische) Konzepte: Verdrängung/ Durcharbeiten.
(Plot-Ebene)}
