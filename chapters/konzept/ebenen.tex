Das Spiel wird auf verschiedenen Ebenen stattfinden.  
Zum einen ist da die \textbf{Char-kter-Ebene}, auf der die Spieler*innen schlicht ihren 
Wünschen, Begierden, persönlichen Zielen etc. nachgehen: Liebschaften, Geldnöte, 
Kariere, Politik und sofort (hier gehört z.B.~auch der geplante Mord zu).  
Die Charakter-Ebene, wird gleichzeitig überschritten durch die \textbf{Zeit-Ebene},
in der bestimmte Handlungen und Konflikte sich \emph{durch die verschiedenen Zeiten
hindurch} fortführen und ergänzen. (Die Zeit-Ebene könnte auch als Familien- oder
Stammbaum-Ebene bezeichnet werden, da sie die Anteile der Charakter-Ebene
bezeichnet, die sich über die individuellen Charaktere durch die Zeiten hindurch
fortsetzen.) Beide Ebenen können (und werden) sich teils auch in
Gruppenkonstellationen abspielen, diese sind aber immer vor allem durch
persönliche Ziele bestimmt. Es geht also mehr darum, ob Hermine sich an Malfoy
rächen kann, ob Harry mit seiner Mannschaft im Quidditch gewinnt oder ob Ron seine
Eifersuchtskonflikte mit Harry überwindet und nicht ob auf der Seite der
Todesser für Voldemort oder auf der Seite der Order des Phönix gestritten wird.  
Charakter- und Zeit-Ebene bauen direkt aufeinander auf. Quer dazu liegt die
\textbf{Plot-Ebene}, welche die Aufgabe übernimmt, die Spieler*innen im >>bigger
Picture<< zusammenzubringen. Sie überschreitet persönliche oder Gruppen-Interne
Interessen. Der Konflikt auf dieser Ebene wird vor allem durch die zwei Parteien 
\nameref{sec:p1} und \nameref{sec:p2} vor angebracht.   
Noch einmal quer zu allen anderen Ebene liegt die **Spielmechanik-Ebene**. Diese
zeichnet sich vor allem dadurch aus, dass sie zunächst \emph{keiner} Spieler:in
bewusst ist. Die Spielmechanik-Ebene kann potenziell allen Parteien und allen
persönlichen Zielen zum Verhängnis werden: Sollten zu viele Zeitreisen
durchgeführt worden sein, tritt die Katastrophe ein. 

Die Ebenen lassen sich kurz im Grad des Bewusstseins durch die Spieler*innen
zusammenfassen:

\begin{itemize}
  \item Die \textbf{Charakter-Ebene} ist jede Spieler:in bewusst. \emph{(Ich/
    wir gegen einige)}
  \item Die \textbf{Zeit-Ebene} speist sich direkt aus der \emph{Charakter-Ebene} und wird den
    Spieler*innen bewusst, sobald sie mit den anderen Zeiten in Berührung kommen,
    indem sie ihre Charakter-Ziele in den Spieler*innen aus den anderen Zeiten
    wiedererkennen. \emph{(Ich/ wir gegen alle)}
  \item Die \textbf{Plot-Ebene} ist nur einigen wenigen Spieler*innen bewusst (vor allem
    denjenigen die zu \nameref{sec:p1} und \nameref{sec:p2} gehören. \emph{(Wir
    gegen die)}
  \item Die \textbf{Spielmechanik-Ebene} ist (zunächst) \emph{niemandem}
    bewusst. \emph{(Alle gegen das Spiel)}
\end{itemize}

Natürlich sind die Ebenen nicht vollständig voneinander getrennt: Die
\emph{Plot-Ebene} kann nur funktionieren, wenn die beiden Parteien
\emph{Anknüpfungspunkte} finden in den individuellen Charakteren. Die
\emph{Spielmechanik-Ebene} kann nur gut in das Spiel integriert sein, wenn sie
sich ein Stück weit aus den anderen Ebenen erklärt, oder plausibilisiert.


