\documentclass[12pt, a4paper, openany]{book}
\usepackage{marginnote}
\usepackage[left=2.5cm,top=3cm, bottom=3cm, right=4cm, marginparwidth=3cm,
marginparsep=0.4cm]{geometry}

% my custom stlye and functions stuff
\usepackage{mystyle}
\usepackage{nameref}

\pagestyle{fancy}
\fancyhf{}
\lhead{kawa-i}
\chead{\glqq Light\grqq}
\rhead{\thepage}

\title{
  { \Large 
    \textbf{\textsc{>>Light<<}}\\
  }
  \vspace{0.4cm}
  { \large \color{gray}
    \textsc{Ein Bildungs-Liverollenspiel}
  }
}
\author{ananym, fux, louis}
\date{{\small \today}}

\begin{document}
\frontmatter
\maketitle

\chapter*{Einleitung}
Am Samstag, den 14. Februar 1968 trifft sich eine Gruppe Jugendlicher auf dem >>Eros
Fest<< (ein lokales Weinfest, dass in den kalten Wintermonaten Dunkelheit
vertreiben und wärme in die Gemüter der Menschen zaubern soll, siehe 
\nameref{ch:offene_fragen} um gemeinsam einen verheißungsvollen Mord zu
begehen.\\\\
%
Das (Bildungs-)Liverollenspiel >>Light<< spielt parallel in den Jahren 1968,
2012 und 2066, an der die Menschen den schicksalsträchtigen Ort besuchen, teils
um sich schlicht auf dem Fest zu berauschen, teils um den (traumatischen)
Erinnerungen von 1968 erneut zu begegnen.\\\\
%
Doch es stellt sich heraus, dass der Mord noch weitreichendere Folgen haben
wird. Durch ihn eröffnet sich nämlich ein Tunnel durch die Zeit, der das
Reisen zwischen den drei Jahren erlaubt, außerdem entzündet sich an ihm
der Streit zwischen zwei rivalisierenden Parteien, die das Zeitreisen für die
Durchsetzung ihrer Zwecke nutzen wollen. Im Laufe des Spiels wird noch eine
dritte Partei auf den Plan treten, die dafür sorgen muss, dass das exzessive
Reisen durch die Zeit nicht noch größeren Schaden anrichten wird, als das
Spielen mit den Naturmächten ohnehin schon hervorbringt: die Katastrophe!


\tableofcontents

\mainmatter
\chapter{Konzept}
\input{chapters/konzept/konzept}
\section{Ebenen}
Das Spiel wird auf verschiedenen Ebenen stattfinden.  
Zum einen ist da die \textbf{Char-kter-Ebene}, auf der die Spieler*innen schlicht ihren 
Wünschen, Begierden, persönlichen Zielen etc. nachgehen: Liebschaften, Geldnöte, 
Kariere, Politik und sofort (hier gehört z.B.~auch der geplante Mord zu).  
Die Charakter-Ebene, wird gleichzeitig überschritten durch die \textbf{Zeit-Ebene},
in der bestimmte Handlungen und Konflikte sich \emph{durch die verschiedenen Zeiten
hindurch} fortführen und ergänzen. (Die Zeit-Ebene könnte auch als Familien- oder
Stammbaum-Ebene bezeichnet werden, da sie die Anteile der Charakter-Ebene
bezeichnet, die sich über die individuellen Charaktere durch die Zeiten hindurch
fortsetzen.) Beide Ebenen können (und werden) sich teils auch in
Gruppenkonstellationen abspielen, diese sind aber immer vor allem durch
persönliche Ziele bestimmt. Es geht also mehr darum, ob Hermine sich an Malfoy
rächen kann, ob Harry mit seiner Mannschaft im Quidditch gewinnt oder ob Ron seine
Eifersuchtskonflikte mit Harry überwindet und nicht ob auf der Seite der
Todesser für Voldemort oder auf der Seite der Order des Phönix gestritten wird.  
Charakter- und Zeit-Ebene bauen direkt aufeinander auf. Quer dazu liegt die
\textbf{Plot-Ebene}, welche die Aufgabe übernimmt, die Spieler*innen im >>bigger
Picture<< zusammenzubringen. Sie überschreitet persönliche oder Gruppen-Interne
Interessen. Der Konflikt auf dieser Ebene wird vor allem durch die zwei Parteien 
\nameref{sec:p1} und \nameref{sec:p2} vor angebracht.   
Noch einmal quer zu allen anderen Ebene liegt die **Spielmechanik-Ebene**. Diese
zeichnet sich vor allem dadurch aus, dass sie zunächst \emph{keiner} Spieler:in
bewusst ist. Die Spielmechanik-Ebene kann potenziell allen Parteien und allen
persönlichen Zielen zum Verhängnis werden: Sollten zu viele Zeitreisen
durchgeführt worden sein, tritt die Katastrophe ein. 

Die Ebenen lassen sich kurz im Grad des Bewusstseins durch die Spieler*innen
zusammenfassen:

\begin{itemize}
  \item Die \textbf{Charakter-Ebene} ist jede Spieler:in bewusst. \emph{(Ich/
    wir gegen einige)}
  \item Die \textbf{Zeit-Ebene} speist sich direkt aus der \emph{Charakter-Ebene} und wird den
    Spieler*innen bewusst, sobald sie mit den anderen Zeiten in Berührung kommen,
    indem sie ihre Charakter-Ziele in den Spieler*innen aus den anderen Zeiten
    wiedererkennen. \emph{(Ich/ wir gegen alle)}
  \item Die \textbf{Plot-Ebene} ist nur einigen wenigen Spieler*innen bewusst (vor allem
    denjenigen die zu \nameref{sec:p1} und \nameref{sec:p2} gehören. \emph{(Wir
    gegen die)}
  \item Die \textbf{Spielmechanik-Ebene} ist (zunächst) \emph{niemandem}
    bewusst. \emph{(Alle gegen das Spiel)}
\end{itemize}

Natürlich sind die Ebenen nicht vollständig voneinander getrennt: Die
\emph{Plot-Ebene} kann nur funktionieren, wenn die beiden Parteien
\emph{Anknüpfungspunkte} finden in den individuellen Charakteren. Die
\emph{Spielmechanik-Ebene} kann nur gut in das Spiel integriert sein, wenn sie
sich ein Stück weit aus den anderen Ebenen erklärt, oder plausibilisiert.



\section{Setting}\label{sec:setting}
Menschen, die alle mehr oder weniger mit dem Mord im Jahre 1968 verknüpft sind,
treffen sich 1968, 2012 und 2066 auf dem (Eros-) Fest an dem der Mord
stattfand.\\\\
%
Die Welt unterscheidet sich im Prinzip nicht von der Welt, wie sie 1968, 2012
und 2066 war, ist, bzw. sein wird, außer dass es die Möglichkeit gibt zwischen
diesen Zeiten zu reisen.\\\\
%
Nach 1968 wurde auf dem Eros-Fest (Samstagmittag 12:00?) auch de\note{Erster Erfolg der Verdrängen-Gruppe?}
gestorbenen Laura Palma gedenkt. Diese Praxis wurde aber im Laufe der 70er
eingestellt\\\\
%
Nach dem Mord haben sich die >>Durcharbeiten<< und >>Verdrängen<< Gruppe (P1/
P2) ausdifferenziert.\\\\
%
Die Gruppen entdeckten schließlich den ebenfalls beim Mord entstandenen\note{Romeo und Julia} 
Zeittunnel und lernten das Zeitreisen kennen. 
Um für ihre Partei einen Vorteil zu erringen, reisten Vertreter*innen beider
Gruppe zu Zeit der Gründung des Dorfes (1879?) zurück. 
Dort wuchs als Kinder zweier Menschen aus p1 und p2 Laura Palma auf.\\\\
% 
Laura Palma erkannte später (nach 1968), dass sie -- als Kind von Zeitreisenden und
verfeindeter Parteien -- an Mord und Konflikt Schuld sei und reiste selbst in
die Vergangenheit (ins Jahr 1968) zurück, um dafür zu sorgen, dass sie stirbt.
Dadurch erzeugt sie erst den Mord.

\section{Spiel Mechanik}
Durch äußerliche Merkmale (z.B. Choker. Siehe:
\nameref{ch:offene_fragen}), wissen die Spieler:innen, wer zu ihrer Zeit
gehört (erklärt in den Spiel-regeln *ohne* dabei auf die Zeiten einzugehen:
>>z.B.~Menschen mit einer anderen Farbe Chokern, siehst, fühlst und hörst du
nicht (-- du kannst sie aber riechen?)<<.\\\\
%
Zunächst sind die Zeiten zusätzlich \emph{räumlich} getrennt (z.B. auf
unterschiedlichen Stockwerken der Unterkunft, die aufgrund von
Unwetterwarnungen verschlossen sind).\\\\
%
Reisen durch die Zeit wird durch einen Wechsel des Chokers dargestellt.\\\\
%
Es ist nur eine bestimmte Zahl an Zeitreisen erlaubt, dann tritt die
Katastrophe ein. Evtl.~gibt es einen physischen Countdown, der auch entdeckt
werden kann (so könnte z.B.~\nameref{sec:p3}, die Partei der
Zeitwächter*innen, zu Beginn dieses physische >>Gerät<< entdecken, ohne
genau zu wissen, womit sie es zu tun haben).\\\\
%
Die Zahl der erlaubten Zeitreisen bis zum Eintritt der Katastrophe
(Siehe: \nameref{ch:offene_fragen}), sind so begrenzt, dass sie gerade so
reichen, die für gewissen Spielziele \emph{notwendigen} Reisen zu
absolvieren.\\\\
%
Stirbt eine Spieler:in während des Spiels, bekommt sie eine neue Rolle
aus \nameref{sec:zc} (2066) zugewiesen. Um zu vermeiden, dass die erste
Person die stirbt zunächst alleine ist, wird die NSC, welche die ermordete
Person spielt, zu Z-C hinzukommen, sobald die erste Spieler:in stirbt.\\\\
%
\qcert{\nameref{sec:zc} ist entweder auf der Seite von \nameref{sec:p1}, oder
auf der Seite von \nameref{sec:p2} (das kann z.B.~dadurch gelöst werden,
dass bestimmte >>Meilensteine<< bzgl.~der Ziele von P-1 und P-2 dazuführen,
dass die nächste Person, die zu Z-D hinzukommt, d.h.~die stirbt, das Ziel
von entweder P-1 oder P-2 unterstützt.)}\\\\
%
\qcert{Option A: Die >>Katastrophe<< ist das Eintreten der \nameref{sec:zc} in die
anderen Zeiten (Problem: aus bildungspolitischen Gründen, d.h.~aufgrund der
Parallelen zur Klimakrise, sollte die Katastrophe abwendbar sein, was das
Eintreten von Z-D optional machen würde). Option B: Z-C kommt zu einem fixen
Zeitpunkt zum Spiel dazu.}\\\\

\section{Bildungsinhalte}
Die Katastrophe steht symbolisch für den Klimawandel, da sie durch exzessives
Nutzen von >>schädlicher<< Technik (zu viele Zeitreisen) ausgelöst wird. 
Wie auch der Klimawandel muss das Faktum der drohenden Katastrophe zunächst
entdeckt und dann überhaupt an alle (Menschen unterschiedlichster Interessen --
und unterschiedlicher Zeiten) vermittelt werden.\\
Wie auch der Klimawandel, muss die Katastrophe durch gemeinschaftliches Handeln,
gegen die unmittelbaren persönlichen Interessen, durchgesetzt werden.
(Spielmechanik-Ebene)\\\\
%
Die drei Zeiten sollen unterschiedliche Stufen der feministischen Bewegung
thematisieren (first, second, third wave feminism). In \nameref{sec:zc}, ist der
Feminismus in der Auflösung der Geschlechts-Identitäten aufgehoben.
(Charakter-Ebene)\\\\
%
Durch das Zusammenführen der Zeiten geraten (auch) die divergierenden
politischen Ansichten in Konflikt. (Zeit-Ebene)\\\\
%
\qcert{Der Kampf zwischen \nameref{sec:p1} und \nameref{sec:p2} thematisiert
verschiedene (psychoanalytische) Konzepte: Verdrängung/ Durcharbeiten.
(Plot-Ebene)}

\section{Kollabiertes Zeitfenster}\label{sec:zeitfenster}
Unangenehmer Raum, Videos, einstürzende Neubauten, Noise, kleine Rätsel (an den
Computer) aktivieren nächste Sequenz, erst die letzte Sequenz öffnet die
Tür.\\\\
%
Dieser Raum kann mehrfach genutzt werden: 
\begin{itemize}
  \item Für die Zeitwächter*innen zu Beginn des Spiels 
  \item Jedes mal, wenn eine Zeitreise durchgeführt wird 
  \item Wenn ein Charakter stirbt und bevor er in \nameref{sec:zc}
    wechselt.
\end{itemize}


\chapter{Main Plot}
\emph{Der Main-Plot dreht sich hauptsächlich um alles was zur \textbf{Plot-Ebene}
gehört, d.h.~der Konflikt zwischen P-1 und P-2. 
Er lässt sich aber nur konzeptionell von den anderen Ebenen (vor allem der
Charakter-Ebene) abtrennen. Im Erlebnis der Spieler*innen sollte er nicht als
ein losgelöstes Element wahrgenommen werden. 
Auch die Spielmechanik-Ebene gehört hierher: der Countdown und die drohende
Katastrophe, die den beiden Gruppe zunächst \emph{nicht} bewusst ist.\\
Der Main-Plot ist bisher noch am wenigsten ausgearbeitet.}\\\\
%
Die Feindschaft dieser beiden Gruppen entzündete sich, wie bereits in
\nameref{sec:setting} angemerkt an dem Mord an Laura Palma im Jahre 1968 und
drück sich zunächst in einem anderen Umgang mit dem Mord aus. Während
\nameref{sec:p1} den Tod Laura Palmas bearbeiten und im Gedächtnis des Dorfes
behalten will, strebt \nameref{sec:p2} an, ihn aus dem kollektiven Bewusstsein
zu verdrängen.
\emph{Wichtig ist, dass zunächst nicht klar ist, was mit Laura Palma passiert
ist: ist sie verschwunden, tot, oder tatsächlich ermordet worden?}

Der Main-Plot sollte aber irgendwie damit verbunden sein, dass die Parteien in
die Zeit zurück an die Gründung des Dorfes gereist sind, um Einfluss auf die
Entwicklung des Dorfes zu nehmen. Laura Palma ist das Kind dieser
Zeitreisenden.

Die dirtte Ebene ist, dass Laura Palma selbst in die Zeit zurückgereist ist und
die >>nihilistischen<< Jugendlichen -- die so oder so einen Mord geplant hatten
-- angeregt hat, Laura Palma (d.h.~die jüngere Version ihrer selbst)
umzubringen, weil Laura Palma der Überzeugung war, dass sie als
Kind-das-nicht-hat-sein-sollen, der Grund für die nicht-endenden Konflikte
ist.\\\\

\emph{To be continued...}


\chapter{Side Plots}
\input{chapters/side-plots}

\chapter{Zeiten}
Die Zeiten sollen (auch) verschiedene Phasen oder >>Etappen<< der feministischen
Bewegung widerspiegeln. Angefangen bei den 68’ern, und schließlich 2066, wo à la
Ursula K.~Le Guin der Feminismus in der Auflösung der Geschlechter Identitäten
aufgehoben ist.

\section{Z-A: 1968}\label{sec:za}
Hier ist der Mord geschehen. Die Charaktere aus Z-A sind demnach entweder selbst
an dem Mord beteiligte, oder Person, die sich irgendwie zu ihm verhalten müssen.
Die Konflikte drehen sich demnach (außer >>arbiträrer<< individueller Konflikte),
um die unmittelbare Reaktion/ Bearbeitung des Mordes.

\section{Z-B: 2012}\label{sec:zb}
Die Menschen aus Z-B spielen (zum großen Teil) ältere >>Versionen<< der Menschen
aus Z-A. Für sie ist es die 44ste.~Jährung des Mordes, der in ihnen vermutlich
vor allem Trauer, Wut, Scham, Reue auslöst.
In dieser Zeit wird das erste Mal wieder dem Mord gedacht. D.h.~dass für einige
Emotionen wieder hochkommen, andererseits viele schlicht Vorfreude auf das Fest
empfinden, der Tod Laura Palmas ihnen nicht in Erinnerung ist.

\section{Z-C: 2066}\label{sec:zc}
Der Mord ist hier nur noch wenigen in Erinnerung. Kaum jemand wird ihn noch
miterlebt haben. Er haftet eher wie eine Legende an dem Ort und dem
>>Eros-Fest<<
und spukt im Flüstern des eisigen Februar-Windes.


\chapter{Parteien}
Nicht alle Charaktere sind (von Anfang an) Teil einer der folgenden Parteien.
Die >>Parteien<< bilden eher etwas wie rivalisierende Weltanschauungen mit
entgegenstehenden Zielen. Die meisten Parteien streben allerdings an, die
>>normalen<< Spieler*innen von ihrer Position zu überzeugen.
\begin{itemize}
  \item[] Die Parteien (auch die Zeitwächter*innen) leben in allen drei Zeiten
    und sie alle >>normaler<< Teil des Dorfes.
\end{itemize}
Die Zugehörigen zu P-1 und P-2 (zu Beginn jeweils 2-3 Spieler:innen), wissen als
(vermutlich) einzige, dass Zeitreisen möglich ist. Sie besitzen außerdem jeweils
eine Zeitreisemachine, die aber nur noch >>Treibstoff<< für jeweils 1-2 Zeitreisen
haben. Durch das Lösen verschiedener Aufgaben (entweder mache X, um an
Treibstoff zu kommen, oder mache Y, um Z zu erreichen, wobei du zufällig auf
Treibstoff stoßen wirst), können die Parteien an neuen Treibstoff für mehr
Zeitreisen kommen.

Die Konflikt(e) und Ziele von P-1 und P-2 sind noch nicht näher erdacht. Sie
sollen aber spannend, innovativ (nicht einfach Gut gegen Böse) und vor allem gut
in die Charakter-Ebene eingeflochten sein (z.B.~mit dem Mord verknüpft).

\section{P-1: >>Durcharbeiten<<}\label{sec:p1}
Die zentralen Ziele dieser Partei lauten (bisher):
\begin{itemize}
  \item Es muss wie in allen Folgejahren auch 1968 eine Trauerfeier stattfinden
  \item Wie in allen Folgejahre auch, muss sie ebenfalls um z.B.~12:00 stattfinden
\end{itemize}

\section{P-2: >>Verdrängen<<}\label{sec:p2}
Die zentralen Ziele dieser Partei lauten (bisher):
\begin{itemize}
  \item Es darf auf gar keinen Fall eine Trauerfeier stattfinden, sonst würde
    die Dorfeinheit zerbrechen/ Rachemorde etc.
\end{itemize}

\section{P-3: Zeitwächer*innen}\label{sec:p3}
Die Zeitwächter*innen haben als einzige Andeutungen zur Katastrophe.\\\\
%
Eine Idee war das die Zeitwächter:innen zu Beginn des Spieles\note{ZW näher an P-2?} 
zusammen sind und durch eine kleine Reihe an Aufgabe erledigen müssen, um
dadurch erst Informationen über das Ende der Zeitreisen zu erfahren. Diese
Aufgaben erledigen sie im >>\nameref{sec:zeitfenster}<<.


\chapter{Charaktere}
\input{chapters/charaktere}

\chapter{Regeln}
\input{chapters/regeln}

\chapter{Random Ideen}
\begin{itemize}
  \item twin peaks melodie bei zeitreise
  \item Laura Palmer Anagram
  \item \qcert{Wenn die Trauerfeier stattfindet, tritt Laura Palma mal heilender
    Engel wieder auf (oder sieht fährt als Lucifer auf die Welt herunter)}
  \item Potentielle Katastrophe: eine ganz andere Laura Palma tritt auf
  \item Z-C ist nicht unbedingt auf Seite von P-1 und P-2, sondern auf der Seite der Zeitwächter:innen, bzw. gegen sie
\end{itemize}


\chapter{Offene Fragen}\label{ch:offene_fragen}
\begin{itemize}
  \item Was ist das äußerliche Merkmal, durch welches die Zeiten voneinander getrennt sind?
  \item Wie viele Zeitreisen sind möglich, bis die Katastrophe eintritt?
  \item Zeit 4, kommt Zeit 4 mit Katastrophe oder zu einem fixen Zeitpunkt ins Spiel
\end{itemize}


\end{document}
